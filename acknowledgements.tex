%!TEX root = thesis.tex

At the beginning of this thesis, instead of having a dedication, I choose to have the meaning of a word which I believe describes me the best: \textit{cosmopolitan}.
This word means a lot to me because I love traveling and meeting new people, getting involved in international projects and exchanges.
That is how I got to meet some of the most interesting people I know, and why thanks to them, I want to expand my horizon even more and seek out new experiences.

Networking for me does not only mean connecting computers, but getting to know new people, either in a hot Finnish sauna in Helsinki, at a pub in Dublin, or at a convention center in Milan.
\newline
\newline
\indent
I would like to thank my supervisors for their continued support and encouragement: Prof. Claudio Palazzi and Prof. Pietro Manzoni.
Also thanks to Samu Varjonen, Ashwin Rao, Andrew Rebeiro Hargrave and ‪Naser Hossein Motlagh‬, who are Professors and Researchers at the University of Helsinki.
They have provided me with the MegaSense device and part of the hardware to develop this project.

I offer my sincere appreciation for the learning opportunities provided by all of them, both as teachers, in Padova and Helsinki, and as supervisors.
\newline
\newline
\indent
``\textit{Perfect is the enemy of good}'',  and this thesis is surely not perfect, but I would like to consider it as a stepping stone for what comes next.
It has made me dive into a broader view on LoRa communication and has made me connect with people that otherwise I would have not had a chance to interact with.