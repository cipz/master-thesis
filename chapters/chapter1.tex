%!TEX root = ../thesis.tex

% https://m.imdb.com/name/nm2931325/quotes
\begin{savequote}[65mm]
	 Ideas are common.\\
	 Everybody has ideas.\\
	 Ideas are highly, highly overvalued.\\
	 Execution is all that matters.
	\qauthor{Casey Neistat}
\end{savequote}

\chapter{Introduction}\label{chapter:introduction}

	% https://lucaspolo.eu/isaac-asimov-predicting-todays-internet-1988/
	% https://aminotes.tumblr.com/post/4994624354/isaac-asimov-predicted-the-internet-of-today-20
	
	In an interview from 1988 about school education and its relationship with the Internet, Isaac Asimov, one of the 20$^{th}$ century's most prolific science fiction writers, said ``\textit{now, with the computer, it's possible to have a one-to-one relationship for the many. Everyone can have a teacher in the form of access to the gathered knowledge of the human species}''.
	
	At the time of writing, many things have changed since that interview: people now live in an \textit{hyperconnected world}, where accessing Internet is considered a right, so that everyone has the same learning opportunities.
	
	According to Cisco's 2018 - 2023 Internet Report, ``the number of devices connected to IP networks will be more than three times the global population by 2023'' \cite{cisco1}.
	Authors also predict there will be 29.3 billion networked devices by 2023, a major increase from 18.4 billion in 2018. 
	% https://www.cisco.com/c/en/us/solutions/executive-perspectives/annual-internet-report/air-highlights.html
	Of al these, 50\% is represented by Internet of Things, or IoT, devices.
	
	As stated in one of Forbes's insights, ``\textit{IoT is ranked as the most important technology initiative by senior executives; more important than artificial intelligence and robotics, among many others}'', while, from an economical point of view, ``\textit{of all emerging technologies, the Internet of Things (IoT) is projected to have the greatest impact on the global economy}'' \cite{forbes}.
	The number of devices that are being connected to the Internet is growing day by day and the industry is at its highest peeks.
	
	This new paradigm of computer science can be considered an enabler for the sciences that need large amount of data for creating algorithms and offering better services and more well tailored products.
	The ubiquity of mobile technology has opened the door to a new era of mobile sensing. 
	
%	Starting from the Middle Ages, through the Industrial Revolutions and arriving to the Modern Era, water and air pollution have haunted people all around the world.
%	From early on, in Europe, unsanitary urban conditions favoured the outbreak of population-decimating epidemics of disease, from plague to cholera and typhoid fever.
%	% https://www.history.com/news/the-killer-fog-that-blanketed-london-60-years-ago
%	Later, with the advent of the steam engine in between of the 18th and 19th century, pollutants from factories started to spread in the air, causing problems such as smog and, in some cases, even acid rain, like the ``London Fog'' \cite{fog} in 1952.
	
	Using the data gathered until the second half of the $20^{th}$ century, computer simulations showed how the rise in CO2 levels that can lead to climate change and cause global temperatures to rise steadily in the years.
	Scientists started to take a more common approach to pollutants and climate change by developing instruments capable of more accurate readings and by better understanding how to analyze the gathered data.
	In the preface of 1981's ``\textit{The Design of Air Quality Monitoring Networks}'', the author states that ``\textit{the number of publications in the environmental area is increasing exponentially}'' \cite{airqualitynetworks}.
	This leaves one to think about the state of this research area up to date.
	
%	From the point of view of Computer Science, technology has also evolved exponentially, confirming the validity of Moore's law and increasing the accuracy of sensors and other analog peripherals used to gather data from the environment.
%	In combination with the fact that computers have also become smaller, portable and personal, a new paradigm of digital devices has emerged: the Internet of Things.

	Many publications have been studying the development of low-cost devices, focused on analyzing the quality of the surroundings of individual. 
	
	This thesis takes a focus on a particular project, MegaSense, a personal air quality device developed by the University of Helsinki.
	
	In this thesis, the architecture for connecting such devices is expanded with the use of a LoRa mesh network.
	
	\section{Contributions}\label{sec:contributions}
	
		The main contribution made in this thesis is the creation of a mesh network composed by microcontrollers which communicates via LoRa technology.
		This mesh has been designed to accommodate the message exchange between devices such as air quality sensors, particularly MegaSense, an air quality sensing device from the University of Helsinki.
		
		Air quality sensing, as previously mentioned, is one of the applications of IoT, and this mesh has been designed keeping in mind the communication needed in between these set of devices.
		
	\section{Document outline}\label{sec:document_outline}
		
		This document follows an hourglass structure by first explaining in general the necessary notions to understand the work done and then going more in detail.
		The content is organized as such in the following chapters:
		\begin{enumerate}
			% \setcounter{enumi}{1}			
			\item \hyperref[chapter:introduction]{\textit{Introduction}}: introductory chapter;
			\item \hyperref[chapter:background]{\textit{Background}:} introduces concepts necessary to understand the project and why it has been developed;
			\item \hyperref[chapter:technologies]{\textit{Technologies}}: explains underlying technologies, starting
			from the general definition of a network, common network architectures, radio technologies and micro-controllers;
			\item \hyperref[chapter:related_work]{\textit{Related work}}: described related projects from which project of this thesis has drawn inspiration;
			\item \hyperref[chapter:proposed_solution]{\textit{Proposed solution}:} describes in depth the mesh network developed;
			\item \hyperref[chapter:results]{\textit{Results and experimentation}:} details the experiments and the results made to test the mesh;
			\item \hyperref[chapter:conclusions]{\textit{Conclusions}}: concluding chapter which revisits the concepts of this project and reflects on the possibilities to ameliorate it.
			
		\end{enumerate}