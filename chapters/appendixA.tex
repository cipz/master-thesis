%!TEX root = ../thesis.tex

\renewcommand\thechapter{A}
\chapter{User Manual}\label{ch:AppendixA}
	
	
	
% TODO spostare le cose commentate in capitolo 5
%	The author has decided to add this appendix to the thesis as a small documentation on how to interface and program the pycom boards.
%	
%	The main documentation can be found at % https://docs.pycom.io/
%	
%	In particular this appendix contains the instructions on how to program the Fipy and pytrack, which, as described in [reference to section where I described the used hardware], are the components that the author has used in this thesis
%	
%	Online forum 
%	% https://forum.pycom.io/category/24/getting-started
%	
%	Youtube channel
%	% https://www.youtube.com/c/Pycom/videos
%	
%	Github
%	% https://github.com/pycom/pycom-libraries
%	
%	\section{Requirements}
%		
%		\subsection{Hardware}
%		
%			In order to interface and program the FiPy it is important to 
%			
%			insert image of uart device
%			% https://docs.pycom.io/gettingstarted/programming/usbserial/
%			% https://www.circuitbasics.com/basics-uart-communication/
%			
%			alternatively it can be programmed via the pybytes platform
%			% https://pycom.io/new-pymesh-on-pybytes-video-tutorial/
%		
%			How to configure antennas for the 
%			% https://pycom.github.io/pydocs/gettingstarted/connection/fipy.html#third
%		
%		\subsection{Software}
%		
%			note to update firmware
%			% https://pycom.github.io/pydocs/pytrackpysense/installation/firmware.html
%			% https://core-electronics.com.au/videos/pycom-pytrack-and-pysense-how-to-update-firmware
%			
%			% https://cloud.google.com/architecture/tracking-assets-with-iot-devices-pycom-sigfox-gcp
%			
%			
%			clear flash memory
%			% https://forum.pycom.io/topic/2620/error-when-importing-pysense-pytrack/6
%			% https://docs.pycom.io/gettingstarted/programming/safeboot/
%			
%	\section{Programming the FiPy}
%	
%		\subsection{Hello world}
%		
%		
%		\subsection{Where am I?}
%			% https://docs.pycom.io/datasheets/expansionboards/pytrack2/