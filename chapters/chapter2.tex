%!TEX root = ../thesis.tex

\chapter{Background}\label{chapter:background}

	This chapter introduces the concept of Internet of Things, which is at the base of this thesis, it describes the problem statement and gives an overview on the background work and state of the art present at the moment of writing.
	A deeper understanding of the related work can be found in chapter Related Work.
	
	\section{Internet of Things}
	
 		The \textit{Internet of Things}, abbreviated with \textit{IoT}, has a longer history than many people know about.
 		
 		% Slide 10 corso internet of things helsinki 
 		
% 		Universal Product Code (UPC)
% 		• Developed in 1971 by George Laurer at IBM
% 		• Still in widespread use
% 		• Adoption relied on emergence of laser optics,
% 		developed around the same time, as these
% 		offered compact reading technology
% 		• Printers used to generate barcodes were
% 		vulnerable to smudge è design coped with errors as ink bleeding would result in taller
% 		bars
% 		• Offered the first way to track products and
% 		address them!

%		Slide 11
%		
%		• 1980 – mid 90s: connecting everyday objects
%		• CMU coke machine often quoted
%		as the first IoT device
%		• Sensors used to detect whether
%		shelves have bottles
%		• Simple algorithms used to track
%		status of coke bottles (warm, cold,
%		empty)
%		• Allow remote access to check
%		status of machine
%		• Communication took place through
%		Arpanet at CMU as the system
%		predated the Internet

%	https://www.smithsonianmag.com/innovation/kevin-ashton-describes-the-internet-of-things-180953749/
%	http://www.itrco.jp/libraries/RFIDjournal-That%20Internet%20of%20Things%20Thing.pdf
%	https://www.rfidjournal.com/that-internet-of-things-thing

	The term ``\textit{Internet of Things}'' which is now know all around the globe, has been attributed to \textit{Kevin Ashton}, who used it in a presentation at \textit{Protector \& Gamble} in 1999 \cite{iot_definition} to describe the network connecting objects in the physical world to the Internet.
	
	\section{Air quality}

		% Describe the problem with air quality, link some papers that show that it has been getting worse in the last decades
		% Show that there are standards to measure air quality		
		% Say something about air quality in the covid era (covid has helped people re-evaluate the use of bikes and other publicly-shared methods of transportation)
		% Stopping, although painful for the society (physically and economically) has helped demonstrating that there is still hope to get a clean environment in which to live
	
	% TODO mantenere le sezioni succesive in questo capitolo o spostare nel successivo?
	
	\section{Solutions to detect air pollution}
	
		% Describe these solutions, history, pricing (from the ones that cost a lot and are very reliable to the ones that are cheap and less reliable; a high amount of cheap devices can perform almost as good as a high end device)
		% Describe some solutions that have been previously proposed by other researchears and universities
	
	\section{\megasense}\label{sec:megasense}
		
		% Describe the megasense project, put pictures and explain how it has been developed until now
	
