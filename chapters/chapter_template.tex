%!TEX root = ../thesis.tex

\begin{savequote}[75mm]
This is a quote
\qauthor{This is the author of the quote}
\end{savequote}

\chapter{This is a chapter}\label{ch:chapter_reference}

	Footnote\footnote{footnotes working fine}

	\lipsum[1-2]
	
	\newpage
	
	\section{Section one}\label{sec:section_one_reference}
	
		Let me list some useful commands:
		\begin{itemize}
			\item This is a \Quote{quote}.
			\item This, \Chapref{chapter_reference}, is a chapter reference.
			\item This adds the following \gls{word}\glsadd{word} to the glossary.
		\end{itemize}
		
	%	This is a figure:	
	%	\begin{figure}[H]
	%		\centering
	%		\includegraphics[width=1\textwidth]{resources/legacy-code}\\
	%		\caption{Dilbert on legacy system infrastructures}
	%	\end{figure}
	
	%   This is an image on the left of a text
	%	\noindent
	%	\begin{minipage}{0.5\textwidth}% adapt widths of minipages to your needs
	%		\includegraphics[width=\textwidth]{resources/img/pycom-logo-new-rp1}
	%	\end{minipage}%
	%	\hfill%
	%	\begin{minipage}{0.45\textwidth}\raggedright
	%		Yesterday,\\
	%		all my troubles seemed so far away\\
	%		Now it looks as though they're here to stay\\
	%		Oh, I believe in yesterday.
	%	\end{minipage}
	
	\section{Section two}\label{sec:section_two_reference}
	
		\[
			\left[
			\begin{tabular}{@{\qquad}m{.6\textwidth}@{\quad}}
				\raggedright%
				\textbf{Web addresses in texts} \par
				There's a package called \texttt{url} designed for typesetting web addresses.
				Write \mbox{\texttt{\string\usepackage\string{url\string}}} in your preamble; this will provide the
				command \mbox{\texttt{\string\url}}. This command takes an address for the argument and
				will print it out with typewriter font. Furthermore, it is able to handle
				special characters in addresses like underscores and percent signs. It even
				enables hyphenation in addresses, which is useful for websites with a very
				long name.%
			\end{tabular}
			\right]
		\]
	
	\section{Section three}\label{sec:section_three_reference}
	
		\[
			\left[
			\begin{tabular}{@{\quad}m{.3\textwidth}@{\qquad}m{.5\textwidth}@{\quad}}
				\includegraphics[width=\linewidth]{example-image-a} &
				\raggedright%
				\textbf{Web addresses in texts} \par
				There's a package called \texttt{url} designed for typesetting web addresses.
				Write \mbox{\texttt{\string\usepackage\string{url\string}}} in your preamble; this will provide the
				command \mbox{\texttt{\string\url}}. This command takes an address for the argument and
				will print it out with typewriter font. Furthermore, it is able to handle
				special characters in addresses like underscores and percent signs. It even
				enables hyphenation in addresses, which is useful for websites with a very
				long name.%
			\end{tabular}
			\right]
		\]