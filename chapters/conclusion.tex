%!TEX root = ../thesis.tex

\begin{savequote}[70mm]
	Be fearful when others are greedy,\\
	and greedy when others are fearful.
	\qauthor{Warren Buffet}
\end{savequote}

\chapter{Conclusions}\label{chapter:conclusions}

	This thesis contains the description of a mesh network, Open LoRa Mesh, of low powered devices in which nodes communicate using LoRa technology.
	Such solution has been thought to work with IoT devices, particularly with an air quality sensing device: MegaSense, described in Section~\ref{subsec:megasense}.
	
	It is important to understanding characteristics of pollutants in urban environments for counteracting problems linked to poor air quality and for assessing the effectiveness of initiatives designed for tackling it.
	Open LoRa Mesh helps in the placement of devices across a larger space where connectivity might be an issue.
	
	As said by authors of \cite{Sampaio-2015}, mesh networks are ``\textit{decentralized, easy to deploy, and characterized by dynamic self-organization, self-configuration, and self-healing properties}''.
	Such properties enable fast deployment, low installation cost, and reliable communication.
		
	Implementation of Open LoRa Mesh has been done in Micropython for Pycom hardware, particularly it has been programmed for the FiPy board.
	Software is available on GitHub under the repository \textit{Open LoRa Mesh} \footnote{ \url{www.github.com/cipz/OpenLoRaMesh}}.
	
	This conclusive chapter summarizes the previously discussed arguments in this thesis, presents possible improvements, both hardware and software, and contains some personal considerations by the author.

	\section{Future work}
		
		Open LoRa Mesh can be considered as a proof of concept that shows LoRa's validity for creating a network which allows devices like MegaSense to communicate and exchange useful data, like on updating calibration.
		However, there are various aspects of this project that can be better improved, both from a hardware and a software point of view.
	
		\subsection{Hardware improvements}
		
			In this project, the hardware that composes each node is quite straightforward: a FiPy device with an expansion board, an antennas for LoRa communication and, where needed, an antenna for GPS and, last but not least, the MegaSense.
			
			A possible improvement could be adding a hard 3D printed shell to each node.
			This would contain the FiPy and the Megasense, with openings for antennas and MegaSense's air sensors.
			
			If functionalities such as the SD card and the GPS antenna are not needed, also if nodes have already been flashed with the software and do not need the micro USB port, shields can be removed.
			Without the shield, FiPys can be powered by either a battery or a direct power source that connects to a 5V wall charger.
			
			About powering the device, an external battery, either LiPo (Lithium Polymer) or alkaline cell, can be added when devices are placed in locations harder to reach.
			Upon these, battery sensors, such as the ones in the MegaSense, can be added to evaluate the battery charge and communicate when they are almost out of energy.

		\subsection{Software improvements}\label{sec:software_improvements}
		
			The most important software improvement is on routing table, which could be populated keeping in mind the \textit{Received Signal Strength Indicator} (RSSI) of incoming messages.
			With the use of this indicator a more stable network can be created, since nodes messages would be better routed through more stable connections among nodes.

			Besides the \texttt{DEFAULT\_BOOT} and \texttt{PLAIN\_LISTENER} modes, other modes could be added to the nodes.
			These can include, for example, a relay station which only forwards packets and does not participate in data creation or communication with air quality sensing devices.
			Such nodes can strategically be placed so that nodes hard to reach, because inside a building or really far away from other nodes, can communicate with the rest of the network.

			As explained in Section~\ref{subsec:loop}, there can be Exceptions that terminate the main loop, thus better Exception handling can be added to the software.
			This would allow for a more stable communication between nodes when, for example, transmissions overlap and packets get corrupted.
			Retransmission backoff methods, for a more controlled system, can be added as a possible solution.

			Another factor that would improve the performance of each node on the network is the implementation of a sleep mode, Section~\ref{sec:sleep}.
			This would allow each node to have a longer battery life and only activate when necessary.
			Although adding such mode requires more data and in depth testing, it is not an impossible upgrade.

		\subsection{On LoRa and other transmission technologies}\label{sec:other_transmission_methods}
		
			LoRa is a promising transmission technologies, but it also comes with some disadvantages, such as the low data rate and high transmission time.
			There are other technologies that could be used for a project such as this one, particularly NB-IoT and 5G would allow for transferring higher amounts of data.
			
			They would not allow thought to have a mesh network since they rely on the cellular technology, thus need a cellular tower and a SIM card.
			Wi-Fi, on the other hand, does not supply a substantial range for these devices to work, as proved by the project described in Section~\ref{subsec:ardueco}.
			In this case, LoRa is the most logical choice since it allows devices to rely on themselves.

	\section{Personal considerations}
	
		This final section of the thesis contains some personal considerations from the author regarding the work with Pycom boards and mesh networks.
	
		\subsection{About the Pycom boards}\label{sec:working_with_pycom}
		
			Compared to working with other boards, such as the Arduino, Pycom boards are more complete, since they offer better specifications, for both computing power and memory, connectivity and support.			
			For prototyping, Arduino boards offer very good opportunities and have been on the market for a longer time, allowing consumers to get acquainted and know the brand.
			% https://patentlyo.com/patent/2011/01/tracing-the-quote-everything-that-can-be-invented-has-been-invented.html
			Sometimes, working with Arduino feels like ``\textit{everything that can be invented has been invented}'', as said by Charles H. Duell, since there is so many people who have made all kind of projects and are sharing them online.
			
			The FiPy, on the other hand, offers more connectivity and more power, at the expense of the open source advantages of most of its competitors.
			As explained in Section~\ref{sec:pycom}, besides keeping the blueprints of their boards private, Pycom also has a proprietary firmware on their boards, which sometimes gets in the way of developers that require to fully use the hardware capabilities.
			This also reflects on the software, where some functionalities that are functioning on other boards are not implemented in Pycom's firmware for their devices.
			Nonetheless, the official forum\footnote{ \url{https://forum.pycom.io/}} available for Pycom's boards is full of useful comments and the users are fairly active.
			
			An example where the FiPy would offer a great improvement over an Arduino board would be this smart home application\footnote{ \url{www.instructables.com/Smart-Home-With-Arduino-Ethernet-Shield-and-Teledu}}.
			Such project contains an Arduino Mega board with various sensors and relays, all connected via a specifically build website, with the use of a third-party API handler for the Arduino.
			The Arduino connects to the Internet via an Ethernet shield, thus sends data by cable.
			A FiPy would allow a smaller footprint and no need for a shield, besides avoiding the use of a third-party API handler.
					
			In my personal opinion, these boards are quite powerful and resilient for research purposes.
			They can be used as multiple testbenches for different projects, but might result excessive if their powerful specifications are not necessary.
			
		\subsection{About mesh networking}\label{sec:mesh_considerations}
			
			As said in Chapter~\ref{chapter:technologies}, each network topology is better suited for particular scenarios.
			It is not possible to have a ``\textit{one size fits all}'' network for all possible applications that need to exchange data.
			As for all questions in computer science, the answer to the question where a mesh network is useful is ``\textit{it depends}''.
			
			Mesh networking is a very interesting network topology which, compared to all the others in Figure~\ref{img:network_topologies}, has still a lot of research possibilities.
			In conjunction with AI and machine learning, it is possible to achieve an optimal level of connectivity in such networks.
			The bottlenecks would then be represented by the computing power in the nodes, and in the algorithm used in case one of the nodes fail, or new nodes are added into the network.

			Scalability of such networks, as spoken about in Section~\ref{sec:scalability}, is a very interesting research scenario since it is one of the major deciding elements for any new networking technology to be accepted.
			Factors such as multi-hopping, protocol overhead, and wireless interference limit the scalability of such networks.
			Mesh networks will be more and more common, but I believe they will be used in conjunctions with other topologies: for example a local mesh network that allows communication between sensors in a house and a gateway node interacts with other networks in a different topology.
			
			Security is another important factor when considering in mesh networking.
			Since each node receives and analyzes packages, there is no totally effective solution for a secure network.
			Adding security features such as encryption/decryption algorithms and other checks, would result in a big overhead on small IoT devices.
			Also, like other radio technologies, anyone can read messages, encrypted or not, ``\textit{flying}'' through the air.