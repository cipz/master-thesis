%!TEX root = ../thesis.tex

\begin{savequote}[70mm]
	Be fearful when others are greedy,\\
	and greedy when others are fearful.
	\qauthor{Warren Buffet}
\end{savequote}

\chapter{Conclusions}\label{chapter:conclusions}

	Mesh networks are ``\textit{decentralized, easy to deploy, and characterized by dynamic self-organization, self-configuration, and self-healing properties}'' \cite{Sampaio-2015}.
	Such properties enable fast deployment, low installation cost, and reliable communication.
	
	This thesis describes an implementation of a mesh network solution for IoT devices, particularly for an air quality sensing device: MegaSense, described in Section~\ref{subsec:megasense}.
	
%	This project represents
%	
%	It has been implemented on pycom hardware, the software is available on github in 
	
	Besides summarizing the previously discussed arguments, this chapter contains some personal considerations by the author.

	\section{Contributions}
	
		\textbf{\textcolor{red}{\hl{// To be completed}}}

	\section{Future work}
		
		\textbf{\textcolor{red}{\hl{// To be completed}}}
	
		\subsection{Hardware improvements}
			
%			Note from the author, it must be taken in consideration that the device can run hot, an application of the device such as on a bike that is parked out in the sun may overheat the device.
%			On pycom's documentation it is written that the fipy can support temperatures up to x
%			Thus an interesting improvement could be to add a heatsink in order to lower the temperatures and improve the lifespan of the device
		
			\textbf{\textcolor{red}{\hl{// To be completed}}}

		\subsection{Software improvements}\label{sec:software_improvements}
		
			\textbf{\textcolor{red}{\hl{// To be completed}}}
		
%			Better logging 
%			
%			improvement on routing table and algorithm used for message forwarding 

%			Aggiungere altre modalità oltre a quella di plain listener e del main, ad esempio relay station per fare forwarding dei messaggi

			Another factor that would improve the performance of each node on the network is the implementation of a sleep mode, Section~\ref{sec:sleep}.
			This would allow each node to have a longer battery life and only activate when necessary.
			Although adding such mode requires more data and in depth testing, it is not an impossible upgrade.

	\section{Personal considerations}
	
		This final section of the thesis contains some personal considerations from the author regarding the work with Pycom boards and mesh networks.
	
		\subsection{About the Pycom boards}\label{sec:working_with_pycom}
		
			Compared to working with other boards, such as the Arduino, Pycom boards are more complete, since they offer better specifications, for both computing power and memory, connectivity and support.			
			For prototyping, Arduino boards offer very good opportunities and have been on the market for a longer time, allowing consumers to get acquainted and know the brand.
			% https://patentlyo.com/patent/2011/01/tracing-the-quote-everything-that-can-be-invented-has-been-invented.html
			Sometimes, working with Arduino feels like ``\textit{everything that can be invented has been invented}'', as said by Charles H. Duell, since there is so many people who have made all kind of projects and are sharing them online.
			
			The FiPy, on the other hand, offers more connectivity and more power, at the expense of the open source advantages of most of its competitors.
			As explained in Section~\ref{sec:pycom}, besides keeping the blueprints of their boards private, Pycom also has a proprietary firmware on their boards, which sometimes gets in the way of developers that require to fully use the hardware capabilities.
			This also reflects on the software, where some functionalities that are functioning on other boards are not implemented in Pycom's firmware for their devices.
			Nonetheless, the official forum\footnote{ \url{https://forum.pycom.io/}} available for Pycom's boards is full of useful comments and the users are fairly active.
			
			An example where the FiPy would offer a great improvement over an Arduino board would be this smart home application\footnote{ \url{www.instructables.com/Smart-Home-With-Arduino-Ethernet-Shield-and-Teledu}}.
			Such project contains an Arduino Mega board with various sensors and relays, all connected via a specifically build website, with the use of a third-party API handler for the Arduino.
			The Arduino connects to the Internet via an Ethernet shield, thus sends data by cable.
			A FiPy would allow a smaller footprint and no need for a shield, besides avoiding the use of a third-party API handler.
					
			In my personal opinion, these boards are quite powerful and resilient for research purposes.
			They can be used as multiple testbenches for different projects, but might result excessive if their powerful specifications are not necessary.
			
		\subsection{About mesh networking}\label{sec:mesh_considerations}
			
			As said in Chapter~\ref{chapter:technologies}, each network topology is better suited for particular scenarios.
			It is not possible to have a ``\textit{one size fits all}'' network for all possible applications that need to exchange data.
			As for all questions in computer science, the answer to the question where a mesh network is useful is ``\textit{it depends}''.
			
			Mesh networking is a very interesting network topology which, compared to all the others in Figure~\ref{img:network_topologies}, has still a lot of research possibilities.
			In conjunction with AI and machine learning, it is possible to achieve an optimal level of connectivity in such networks.
			The bottlenecks would then be represented by the computing power in the nodes, and in the algorithm used in case one of the nodes fail, or new nodes are added into the network.

			Scalability of such networks, as spoken about in Section~\ref{sec:scalability}, is a very interesting research scenario since it is one of the major deciding elements for any new networking technology to be accepted.
			Factors such as multi-hopping, protocol overhead, and wireless interference limit the scalability of such networks.
			Mesh networks will be more and more common, but I believe they will be used in conjunctions with other topologies: for example a local mesh network that allows communication between sensors in a house and a gateway node interacts with other networks in a different topology.
			
			Security is another important factor when considering in mesh networking.
			Since each node receives and analyzes packages, there is no totally effective solution for a secure network.
			Adding security features such as encryption/decryption algorithms and other checks, would result in a big overhead on small IoT devices.
			Also, like other radio technologies, anyone can read messages, encrypted or not, ``\textit{flying}'' through the air.