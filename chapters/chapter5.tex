%!TEX root = ../thesis.tex

% https://en.wikipedia.org/wiki/KISS_principle
\begin{savequote}[70mm]
	\textbf{K}eep\\
	\textbf{I}t\\
	\textbf{S}imple\\
	\textbf{S}tupid\\
%	\qauthor{Bill Gates}
\end{savequote}

\chapter{Proposed solution}\label{chapter:proposed_solution}

	\section{Idea}
	
	\section{Architecture}
	
		%Describe sensing pipeline 
	
	\section{Hardware}
	
		%Describe hardware requirements, connect to the sections in chap 3 
		%
		%Altri microcontroller, come esp8266 che sono stati generati da arduino sono stati scartati in quanto poca documentazione, ecc
	
	\section{Software}
	
	% Hardware Addressing Schemes
	
	
	% https://www.circuitbasics.com/basics-uart-communication/
	%UART
	%
	%How I programmed the pycom
	
	% Author's note, I believe that, due to some lack of documentation from the board manufacturer's webpage, I believe it is important to include in this thesis some technical points on how the boards have been programmed (oppure meglio mettere in un'appendice a parte??? Forse meglio perchè in questo caso si vanno a separare i concetti che riguardano direttamente la tesi da quelli che riguardano )
	%
	%
	%calcolare transmission times
	
		The software for this 
		
		\subsection{Algorithms}
	
			% https://latexdraw.com/automata-diagrams-in-latex/
			% https://hayesall.com/blog/latex-automata/
			% https://isaaccomputerscience.org/concepts/dsa_toc_fsm?examBoard=all&stage=all
		
			Finite-state machine
		
			To better understand the algorithms, they are represented using finite state machines.
			For a graphical reason, each state of the automatas is abbreviated, and the full state is described in a table underneath
			
			Visione da generale a specifica di ciasun componente
			
			\subsubsection{Main}
				
				\begin{center}
	
					\begin{tikzpicture}[shorten >=1pt,node distance=2cm,on grid,auto]
					
						% Help grid
						% \draw [help lines] (-1,1) grid (6,-6);
						
						\tikzstyle{every state}=[fill={rgb:black,1;white,10}]
						
	
						\node[state, initial]  			(s_0)                 	{$s_0$}; % START
						\node[state] 					(s_1)	[right of=s_0]	{$s_1$}; % BOOT
						\node[state]					(s_2) 	[right of=s_1]	{$s_2$}; % MESH DEFINITION
						\node[state, accepting]			(s_3) 	[right of=s_2]	{$s_3$}; % MAIN LOOP
						
						\path[->]
						(s_0) edge node {} (s_1)
						(s_1) edge node {} (s_2)
						(s_2) edge node {} (s_3)
						(s_3) edge [loop right] node {} ();
					
					\end{tikzpicture}
					
						\begin{table}[htbp]
						\begin{center}
							\begin{tabularx}{0.85\textwidth}{@{}|Y|Y|Y|@{}} 
								\hline
								State abbreviation & Full state name & State description \\\hline
								$s_{0}$ && \\\hline
							\end{tabularx}
							\caption{Main algorithm fsm description}
							\label{table:fsm_main}
						\end{center}
					\end{table}
	
				\end{center}
	
			\subsubsection{Boot up}
			
			\subsubsection{Mesh initiation}
			
			\subsubsection{Broadcast mesh information}
			
			\subsubsection{Loop}
			
	\section{Use cases}
	
	% TODO ADD TO REFERENCES
	% https://docs.pycom.io/gitbook/assets/lopy4-pinout.pdf