%!TEX root = ../thesis.tex

In the last two decades, technology has evolved exponentially, confirming the validity of Moore's law and allowing devices to become smaller and smaller, to the point where they do not only fit in a pocket but can be placed seemingly everywhere.
Such devices, which are mostly composed of sensors, have the purpose of interacting and gathering data from anything they are connected to, thus rendering such things ``\textit{smart}''.

These devices compose the Internet of Things, or IoT: a new paradigm of digital devices that make up about 21.5 billion interconnected devices in the world, at the time of writing.
Many of the publications regarding IoT have been studying the development of low-cost devices, focused on analyzing the air quality surroundings of the individual. 

This thesis expands the architecture for connecting such devices and describes a mesh low-power wide area network, Open LoRa Mesh, created to accommodate low-powered IoT embedded devices that communicate via LoRa technology.
Such network consists of FiPy boards, by Pycom, that talk with MegaSense, a personal air quality device developed by the University of Helsinki.