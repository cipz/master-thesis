%!TEX root = ../thesis.tex

Negli ultimi vent'anni, la tecnologia si è evoluta esponenzialmente, confermando la validità della Legge di Moore e facendo sì che i dispositivi diventassero sempre più piccoli, al punto da non solo poter essere messi in tasca, ma anche essere posizionati in ogni luogo.
Tali dispositivi, principalmente sensori, hanno lo scopo di interagire e raccogliere dati da qualsiasi cosa a cui sono connessi, rendendo dunque tali oggetti ``\textit{intelligenti}''.

Questi dispositivi compongono l'Internet delle cose, o in inglese Internet of Things (IoT): un nuovo paradigma dei dispositivi digitali che costituisce circa 21.5 miliardi di dispositivi connessi nel mondo, al momento della scrittura.
Molte delle pubblicazioni scientifiche a proposito dell'IoT studiano lo sviluppo di dispositivi a basso costo, con lo scopo di analizzare la qualità dell'aria che circonda gli individui.
Questa tesi tratta di un'architettura per l'interconnessione di tali dispositivi e descrive una rete, Open LoRa Mesh, con topologia mesh di dispositivi a basso costo che comunicano utilizzando la tecnologia LoRa.
Tale rete è composta da dispositivi FiPy, prodotti da Pycom, che comunicano con MegaSense, un sensore personale della qualità dell'aria sviluppato all'Università di Helsinki.