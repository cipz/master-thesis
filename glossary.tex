%!TEX root = thesis.tex

% EXAMPLE CHAPTHER 

\newglossaryentry{word}{
	name={This is a word},
	description={This is how a glossary word seems.
%		\cite{source}
	}
}

%INTRODUCTION

\newglossaryentry{Broadband networking}{
	name={Broadband networking},
	description={In general, broadband refers to telecommunication in which a wide band of frequencies is available to transmit information. Because a wide band of frequencies is available, information can be multiplexed and sent on many different frequencies or channels within the band concurrently, allowing more information to be transmitted in a given amount of time (much as more lanes on a highway allow more cars to travel on it at the same time)
	\cite{broadband}}
}

\newglossaryentry{Low latency}{
	name={Low Latency Communication},
	description={Low latency describes a computer network optimized to process a very high volume of data messages with minimal delay (latency). These networks are designed to support operations that require near real-time access to rapidly changing data
	\cite{low_latency}}
}

\newglossaryentry{Core network}{
	name={Core Network},
	description={The core network of a telecommunication infrastructure represents the services that interconnect the main components that create the backbone, which are usually routers and switches}
}

\newglossaryentry{LTE}{
	name={Long-Term Evolution (LTE)},
	description={Long-Term Evolution is a standard for wireless broadband communication for mobile devices and data terminals}
}

%CHAPTER 2

\newglossaryentry{Roadmap}{
	name={Roadmap},
	description={A roadmap is a strategic plan that defines a goal or desired outcome and includes the major steps or milestones needed to reach it}
}

\newglossaryentry{Repository}{
	name={Repository},
	description={A software repository is a storage location from which software packages, verisoned and well documented, may be retrieved and installed on a computer}
}

\newglossaryentry{Milestone}{
	name={Milestone},
	description={A milestone is a specific point in time within a project lifecycle used to measure the progress of a project toward its ultimate goal.
	Milestones are associated with working deliverables that, in case of future flops, can be used as checkpoints}
}

%CHAPTER 3

\newglossaryentry{Framework}{
	name={Framework},
	description={
		A framework is an abstraction that establishes common practives for creating, interpreting, analyzing and using architecture descriptions within a particular domain of application}
}

%CHAPTER 5

\newglossaryentry{CentOS}{
	name={CentOS},
	description={CentOS is an Enterprise Linux distribution based on the sources from Red Hat Enterprise Linux}
}

\newglossaryentry{Virtual Machine}{
	name={Virtual Machine},
	description={A Virtual Machine (VM) is an emulation of a computer system}
}

\newglossaryentry{Remmina}{
	name={Remmina},
	description={Remmina is a is free and open-source remote desktop client for Linux and other Unix-like systems that allows connecting with other systems from terminal or graphical interfaces}
}

\newglossaryentry{H2 database}{
	name={H2 database},
	description={H2 is an open-source lightweight Java database. It can be embedded in Java applications or run in the client-server mode}
}

\newglossaryentry{Cookie}{
	name={Cookie},
	description={A cookie is a text file that a Web browser stores on a user’s machine. Cookies are a way for Web applications to maintain application state. They are used by websites for authentication, storing website information/preferences, other browsing information and anything else that can help the Web browser while accessing Web servers\cite{cookie}}
}

\newglossaryentry{Context Path}{
	name={Context Path},
	description={The context path is the prefix of a URL path that is used to select the context(s) to which an incoming request is passed\cite{configuring-contexts}}
}

\newglossaryentry{VPN}{
	name={Virtual Private Network (VPN)},
	description={A Virtual Private Network extends a private network across a public network, and enables users to send and receive data across shared or public networks as if their computing devices were directly connected to the private network}
}


%CCONCLUSIONS

\newglossaryentry{Container}{
	name={Container},
	description={A Docker container is an open source software development platform. Its main benefit is to package applications in containers, allowing them to be portable to any system running a Linux or Windows operating system (OS)\cite{docker}}
}

